\documentclass[
	numbers=noenddot, % ¬ .e am Ende von TOC
	toc=flat, %Flache TOC
	12pt, % Schriftgröße 
	titlepage, % es wird eine Titelseite verwendet 
	listof=totoc, % Verzeichnisse im Inhaltsverzeichnis aufführen 
	bibliography=totoc, % Literaturverzeichnis im Inhaltsverzeichnis aufführen 
	a4paper
]{scrartcl}

\usepackage[ngerman]{babel}
\usepackage[utf8]{inputenc}
\usepackage[left=2.5cm,right=2.5cm, top=2cm, bottom=2cm]{geometry}
\usepackage{setspace}
\setlength{\footskip}{1.2cm}  % distance between footer and content
\usepackage[toc,page]{appendix} % Anhang
\renewcommand{\appendixname}{Anhang}
\renewcommand{\appendixpagename}{\appendixname} 
\renewcommand{\appendixtocname}{\appendixname} 

% Document variables
\newcommand{\versiondate}{\today}
\newcommand{\firstAuthor}{Oliver Friedrich}
\newcommand{\secondAuthor}{Johannes Hamfler}
\newcommand{\Authors}{\firstAuthor, \secondAuthor}
\newcommand{\shortAuthors}{O. Friedrich, J. Hamfler}
\newcommand{\Title}{Prototypische Umsetzung eines Chatbots zur Terminverwaltung}
\newcommand{\Subtitle}{Masterprojekt WS 2017}
\newcommand{\Supervisor}{Prof. Dr. rer. nat. habil. Michael Frank}
\newcommand{\Affil}{Hochschule für Technik, Wirtschaft und Kultur Leipzig}
\newcommand{\Faculty}{Fakultät Informatik, Mathematik und Naturwissenschaften}
\newcommand{\unterschriftenfelder}[4]{%
  \parbox{\textwidth}{
    \centering #1, \versiondate\\
    \vspace{2cm}
    
    \parbox{6cm}{
      \centering
      \rule{6cm}{1pt}\\
       #2 
    }
    \hfill
    \parbox{6cm}{
      \centering
      \rule{6cm}{1pt}\\
      #3
    }
  }
}

\usepackage[hidelinks]{hyperref}
\hypersetup{%
    pdftitle={\Title},%
    pdfauthor={\Authors},%
    pdfsubject={\Subtitle},%
}

\usepackage[
    headsepline,%
    footsepline,%
    automark,%
]{scrlayer-scrpage}

\usepackage{xcolor}
% % % % color definitions % % % % 
\definecolor{mylightgreen}{HTML}{DFECCF}
\definecolor{mydarkgreen}{HTML}{95C990}
\definecolor{dkgreen}{rgb}{0,0.6,0}
\definecolor{gray}{rgb}{0.5,0.5,0.5}
\definecolor{lightgray}{gray}{0.5}
\definecolor{mauve}{rgb}{0.58,0,0.82}
\definecolor{darkgray}{rgb}{0.4,0.4,0.4}
\definecolor{purple}{rgb}{0.65, 0.12, 0.82}
\definecolor{orange}{rgb}{1, 0.3, 0}

\usepackage{listings}
\lstset{
  backgroundcolor=\color{white},   % choose the background color; you must add \usepackage{color} or \usepackage{xcolor}
  basicstyle=\footnotesize\ttfamily, % the size of the fonts that are used for the code
  breakatwhitespace=false,         % sets if automatic breaks should only happen at whitespace
  breaklines=true,                 % sets automatic line breaking
  captionpos=b,                    % sets the caption-position to bottom
  commentstyle=\color{dkgreen},    % comment style
  deletekeywords={...},            % if you want to delete keywords from the given language
  escapeinside={\%*}{*)},          % if you want to add LaTeX within your code
  extendedchars=true,              % lets you use non-ASCII characters; for 8-bits encodings only, does not work with UTF-8
  frame=single,                    % adds a frame around the code
  keepspaces=true,                 % keeps spaces in text, useful for keeping indentation of code (possibly needs columns=flexible)
  keywordstyle=\color{blue},       % keyword style
  morekeywords={*,...},            % if you want to add more keywords to the set
  numbers=left,                    % where to put the line-numbers; possible values are (none, left, right)
  numbersep=5pt,                   % how far the line-numbers are from the code
  numberstyle=\tiny\color{darkgray}, % the style that is used for the line-numbers
  rulecolor=\color{black},         % if not set, the frame-color may be changed on line-breaks within not-black text (e.g. comments (green here))
  showspaces=false,                % show spaces everywhere adding particular underscores; it overrides 'showstringspaces'
  showstringspaces=false,          % underline spaces within strings only
  showtabs=false,                  % show tabs within strings adding particular underscores
  stepnumber=1,                    % the step between two line-numbers. If it's 1, each line will be numbered
  stringstyle=\color{mauve},	   % string literal style
  tabsize=2,                       % sets default tabsize to 2 spaces
  title=\lstname,                  % show the filename of files included with \lstinputlisting; also try caption instead of title
}


% yaml
\newcommand\YAMLcolonstyle{\color{red}\footnotesize\ttfamily}
\newcommand\YAMLkeystyle{\color{black}\footnotesize\ttfamily}
\newcommand\YAMLvaluestyle{\color{blue}\footnotesize\ttfamily}
\makeatletter
% here is a macro expanding to the name of the language
% (handy if you decide to change it further down the road)
\newcommand\language@yaml{yaml}
\expandafter\expandafter\expandafter\lstdefinelanguage
\expandafter{\language@yaml}
{
  keywords={true,false,null,y,n},
  keywordstyle=\color{darkgray}\ttfamily,
  basicstyle=\YAMLkeystyle,       % assuming a key comes first
  sensitive=false,
  comment=[l]{\#},
  morecomment=[s]{/*}{*/},
  commentstyle=\color{purple}\ttfamily,
  stringstyle=\YAMLvaluestyle\ttfamily,
  moredelim=[l][\color{orange}]{\&},
  moredelim=[l][\color{magenta}]{*},
  moredelim=**[il][\YAMLcolonstyle{:}\YAMLvaluestyle]{:},   % switch to value style at :
  morestring=[b]',
  morestring=[b]",
  literate =    {---}{{\ProcessThreeDashes}}3
                {>}{{\textcolor{red}\textgreater}}1     
                {|}{{\textcolor{red}\textbar}}1 
                {\ -\ }{{\ttfamily\ -\ }}3,
}
% switch to key style at EOL
\lst@AddToHook{EveryLine}{\ifx\lst@language\language@yaml\YAMLkeystyle\fi}
\makeatother
\newcommand\ProcessThreeDashes{\llap{\color{cyan}\ttfamily-{-}-}}


% enable and define SourceCode-Listings
\usepackage{minted}
\setminted{%
    linenos=true,
    breaklines=true,
    tabsize=4,
    numbersep=5pt,
    fontsize=\footnotesize,
}

%\newmintedfile[ansible]{yaml}{}
%\newmintedfile[vagrant]{ruby}{}

\usepackage[babel,german=quotes]{csquotes}

\usepackage[inkscape=force]{svg}
\usepackage{graphicx}
\graphicspath{{./img/}}
\newcommand{\Logo}{\includesvg[height=30px]{img/HTWK-Logo}}
\newcommand{\BigLogo}{\includesvg[height=60px]{img/HTWK-Logo}}

\usepackage{pdfpages}

\usepackage{tabularx}
%\usepackage{fourier} 
%\usepackage{array}
\usepackage{makecell}
%\renewcommand{\cellalign/theadalign}{tl}
\renewcommand\theadalign{tl}
%\renewcommand\theadfont{\bfseries}
\renewcommand\theadgape{\Gape[0pt]}
\renewcommand\cellgape{\Gape[0pt]}
\setcellgapes{0pt}
%\usepackage{cellspace}
%\setlength\cellspacetoplimit{0pt}
%\setlength\cellspacebottomlimit{0pt}


\usepackage[affil-it]{authblk}
\usepackage[colorinlistoftodos,prependcaption,textsize=tiny]{todonotes}
\setlength{\parindent}{0em}

% For folder listings and trees
%\definecolor{folderbg}{RGB}{255,255,255}
%\usepackage{forest}

\usepackage[acronym,toc,shortcuts]{glossaries}
\makeglossaries%

% \newacronym{DNS}{DNS}{Domain Name Service}

\usepackage[
style=alphabetic,
bibstyle=authoryear,
citestyle=alphabetic,
backend=biber
]{biblatex}
\bibliography{main.bib}

% Enable custom title page
%\titlepage%


\begin{document}


\begin{titlepage}
    \centering
    \textbf{\Huge {\Title}}\\
    \vspace{2cm}
    {\Large \Subtitle}\\
    \vspace{1cm}
    {\Large Betreuer: \Supervisor}\\
    \vspace{2cm}
    {\Large \Authors}\\
    \vspace{2cm}
    {\MakeUppercase{\Affil}}\\
    \vspace{0.5cm}
    {\MakeUppercase{\Faculty}}\\
    \vspace{2cm}
    {\Large vorgelegt am:\\ \versiondate}\\
    \vspace{4cm}
    {\BigLogo}
\end{titlepage}


% % % % % % % % % % % % % % % % % % % % % % % % % INHALTSVERZEICHNIS
\chead{}
\ihead{\footnotesize \textsc{\headmark}}
\ohead{}
\ifoot{\Authors}
\cfoot{}
\ofoot{\versiondate}
\pagebreak

\tableofcontents
\pagebreak
% % % % % % % % % % % % % % % % % % % % % % % % % INHALTSVERZEICHNIS

%%%%%% Verzeichnisse %%%%%%
\pagenumbering{gobble}
\listoffigures %Abbildungsverzeichnis
\newpage
\listoftables %Tabellenverzeichnis
\newpage
\renewcommand\lstlistingname{Listing} % Untertitel unter Listing
\renewcommand\lstlistlistingname{Quellcodeverzeichnis} % Verzeichnisname
\def\lstlistingautorefname{Listing} % autoref im text
\lstlistoflistings % Listingverzeichnis
\newpage
\clearpage
%%%%%%%%%%%%%%%%%%%%%%%%%%%
%\listoftodos



% % % % % % % % % % % % % % % % % % % % % % % % % INHALT
\newpage
\pagenumbering{arabic}
\setcounter{page}{1}
\parindent0pt % Keine Absatzeinrückungen
\parskip2ex %Absatzhöhe

\chead{}
\ihead{\footnotesize \textsc{\headmark}}
\ohead{\thepage}
%\automark[subsection]{subsection}
\automark[section]{section}
\ifoot{\Authors}
\cfoot{}
\ofoot{\versiondate}


\section{Einleitung}
% kurz Abstract
% um was gehts, wer gab den auftrag, welcher hintergrund / motivation
% was ist das grobe ziel

\clearpage
\section{Anforderungen und Festlegungen}
% Ferdi will, Ferdi verlangt, Ferdi lässt Spielraum (feine ziele)
% Festgelegte Begriffe, allgemeingültige Aussagen für das Dokument

\clearpage
\section{Auswahl eines Bot-Frameworks}
% HuBot, BotKit
% Einschränkungen der Auswahl weil ...

\clearpage
\section{Entwurf eines Modells}
% Kommunikationspfade, Blockdiagramme, UML
% Datenhaltung, Datentransport, Sicherheit, Zukunftssicherheit, Austauschbarkeit, Robustheit, Erweiterbarkeit

\clearpage
\section{Realisierung des Modells}
% Warum welche Sprache
% Warum Abkürzungen im ggs zu Modell genommen
% Welche praktischen Hürden überwunden

\clearpage
\section{Analyse}
% Alle Anforderungen erfüllt?
% Wie gemessen?
% Blockdiagramme, MSCs, Fehlerbetrachtung
% Performance - Latenz?

\clearpage
\section{Fazit}
% Funzt? Jo
% Was wurde erledigt, was ¬ und warum?

\clearpage
\section{Aussicht}
% Empfehlungen für Verbesserungen, Erweiterungsmöglichkeiten, ...

\clearpage
\section{Benutzung des Endprodukts}
% Wo Daten eingeben, Wie per Hand testen, Wie Komponenten verbinden, (minimal) Beispiel


%Literatur
\nocite{GitHubGettingStartedHubot}
\nocite{BotkitBotkitToolkitbuilding}
\nocite{SlackConversationsAPI}

% % % % % % % % % % % % % % % % % % % % % % % % % INHALT

\newpage
\begin{appendices}
\section{Erstellung der Datenbank}
%TODO: Name aus Termin-Tabelle nehmen
\lstinputlisting[language=sql, caption={Erstellung der DB}, label=lst:promise]{code/SteckerbotDBModelcreate.sql}
\section{Docker-Umgebung}
\lstinputlisting[language=bash, caption={Dockerfile für Hubot}, label=lst:dockerfile]{code/Dockerfile}
\lstinputlisting[language=yaml, caption={docker-compose.yml}, label=lst:docker-compose]{code/docker-compose.yml}

% \todo{Projektplan dazu?}
\section{Projektplan}
\includepdf[fitpaper]{../docs/Steckerbot_taskinfo.pdf}
\includepdf[fitpaper]{../docs/Steckerbot_ganttchart.pdf}

\end{appendices}

\appendix


%Literatur
% \nocite{heap_ansible:_2016}
\newpage
%\lhead{{\footnotesize \textsc{Quellenverzeichnis}}}
\renewcommand\refname{Quellenverzeichnis}
\printbibliography%

\clearpage

%Abkürzungen ausgeben
%\deftranslation[to=German]{Acronyms}{Abkürzungsverzeichnis}
%\printglossary[type=\acronymtype,style=long,title=Abkürzungsverzeichnis]

%\lstlistoflistings
%\listoffigures


\setcounter{secnumdepth}{0}
\pagebreak
\ihead{\footnotesize \textsc{Selbständigkeitserklärung}}
\ohead{}
\section{Selbständigkeitserklärung}
\large
Hiermit erklären wir, dass wir die vorliegende Hausarbeit selbständig verfasst und keine anderen als die angegebenen Hilfsmittel benutzt haben.
Die Stellen der Hausarbeit, die anderen Quellen im Wortlaut oder dem Sinn nach entnommen wurden, sind durch Angaben der Herkunft kenntlich gemacht. Dies gilt auch für Zeichnungen, Skizzen, bildliche Darstellungen sowie für Quellen aus dem Internet.

\vspace{3cm}
\unterschriftenfelder{Leipzig}{\firstAuthor}{\secondAuthor}

\end{document}
