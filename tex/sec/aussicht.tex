\subsection{Sicherheitsaspekte}

Während der Implementierung des Prototypen wurde gemäß der Anforderungen nicht auf Sicherheitsaspekte eingegangen. Bei weiterem Einsatz des Bot-Protoypen sollten diese aber unter keinen Umständen vernachlässigt werden. Eine Übersicht über zu beachtende Sicherheitsaspekte befindet sich in \autoref{tab:secaspects}. Diese Übersicht stellt dabei keinen Anspruch auf Vollständigkeit, erlaubt aber einen groben Überblick über zu erwartende Sicherheitslücken.

\begin{table}[h]
    \centering
    \begin{tabularx}{\textwidth}{|X|X|X|}
        \hline
        \textbf{Aspekt} & \textbf{Problem} & \textbf{Lösung} \\
        \hline
        Berechtigungen & Adminrechte nicht getrennt & Berechtigungen gemäß DB implementieren \\
        \hline
        Docker-Variablen & in Plaintext gespeichert & Docker Secrets verwenden \\
        \hline
        MySQL-DB & theoretisch anfällig für SQL-Injektion & DB-Eingaben validieren (z.B. RegEx), regelmäßige Backups \\
        \hline
        NodeJS & Sicherheitslücken alter Versionen & regelmäßige Aktualisierungen \\
        \hline
        Passwörter & keine Passwort-Policy vorhanden & nur ausreichend komplexe Passwörter zulassen \\
        \hline
        Verbindungssicherheit & DB-Verbindung beim Anlegen von Nutzern unverschlüsselt & SSL-Zertifikat für Verbindungen, z.B. von Let's Encrypt \\
        \hline
    \end{tabularx}
    \caption{Überblick Sicherheitsaspekte}
    \label{tab:secaspects}
\end{table}

\subsection{Botschnittstelle}
Der Prototyp des Chatbots verfügt über eine rein dialogbasierte Nutzerinteraktion. Jeder Befehl erfolgt über die Textschnittstelle, worüber nur eingeschränkte Interaktionen möglich sind.

Die Implementierung einer erweiterten Schnittstelle in Form eines Slack-Panels\footnote{\url{https://api.slack.com/dialogs}} zur interaktiven Auswahl von Terminen ermöglicht z.B. eine schnellere Auswahl von Terminen, da diese nicht immer einzeln erfragt werden müssen, sondern direkt im Slack-Kanal ausgewählt werden können.
Für diese Funktionalität ist die Registrierung des Bots als \enquote{Slack-App} nötig, da dies nur mit Slack-Webhooks möglich ist.
Der dafür geschätzte Aufwand beträgt 16 h.
