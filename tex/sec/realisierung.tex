\subsection{Datenbank}

Als Datenbank wurde aus den vorher genannten Gründen MySQL ausgewählt.
Für die Erstellung der Datenbank entschieden sich die Autoren für die Verwendung eines Hilfswerkzeugs, in dem das Modell abgebildet und eine Befehlskette ausgegeben wird. Dadurch sollte eine Abweichung der Realisierung vom Modell vermieden und eine rasche prototypische Verifizierung ermöglicht werden. Aktuell (März 2018) existiert kein anerkannter Standard (wie z.B. UML) zur Beschreibung von Datenbanken, weshalb sich die Autoren für das Hilfswerkzeug \enquote{Vertabelo} entschieden. Dieses gibt die notwendigen SQL-Befehle aus, welche direkt für die Erstellung der Datenbank verwendet werden können. Weiterhin können die unter \url{vertabelo.com} angelegten Datenbankmodelle von mehreren Nutzern verwendet werden, was insbesondere für die geplante Weiterverwendung der Datenbank nützlich sein kann.
Die bereits in \autoref{img:db-schema} gezeigte Datenbank ist unter \cite{VertabeloDesignYourDatabase2018} abrufbar. Der daraus generierte SQL-Code befindet sich im Anhang.


\subsection{Containerisierung}

Um das optionale Ziel der Containerisierung zu erreichen, wurde auf einen Standardcontainer der MySQL-Entwickler über Docker zurückgegriffen. Der Container zum Betrieb Hubots basiert auf einem Alpine Linux-Container mit NodeJS, der mittels eines selbst erstellten Dockerfiles erstellt wurde, wie aus \autoref{lst:dockerfile} zu entnehmen.

% Die Autoren entschieden sich gegen den Bau eines eigenen Containers, um die Fehleranfälligkeit zu verringern und dokumentations- und supportkompatibel zu bleiben.

Docker ist eine Technologie zum Betrieb von Anwendungen in Containern, wobei jeder Container seine von den restlichen Containern isolierte Umgebung (Dateien, Abhängigkeiten) enthält, dabei aber die Hardware des Gastsystems mit anderen Containern teilt.

\subsubsection{Vergleich von Containerisierungslösungen}
Neben dem Einsatz der Containerlösung Docker wurden zu Projektbeginn Alternativen entsprechend \autoref{tab:docker-alternatives} betrachtet.

\begin{table}[H]
    \centering
    \begin{tabularx}{\textwidth}{|l|X|X|X|X|}
        \hline
        & \textbf{Docker} & \textbf{nativ} & \textbf{VirtualBox} & \textbf{Cloud} \\
        \hline
        Beschreibung & Umgebung für Linux-Container & lokale Installation der Anwendungen & Vollvirtuali- sierung inkl. Hardware & entfernte, dynamisch einsetzbare Ressourcen \\
        \hline
        Beispiel & docker pull mysql & apt install mysql & bitnami NodeJS VM & \url{heroku.com} \\
        \hline
        Geschwindigkeit     & $++$      & $+++$     & $+$   & $++++$ \\
        \hline
        Datensparsamkeit    & $++++$    & $++$      & $+++$ & $+$ \\
        \hline
        Portierbarkeit      & $++++$    & $+$       & $+++$ & $++$ \\
        \hline
        Persistenz          & $+$       & $+++$     & $++$  & $++++$ \\
        \hline
        \hline
        $\Sigma$            & 11        & 10        & 9     & 11 \\ 
        \hline
        \textbf{Ausschlusskriterium} & - & Testumgebung inhomogen, Serverinfrastruktur unbekannt & instabiles CLI\footnote{command line interface} & Datenschutz nicht gewährleistet, vendor-lock-in, laufende Kosten \\
        \hline
    \end{tabularx}
    \caption{Docker-Alternativen im Vergleich}
    \label{tab:docker-alternatives}
\end{table}

Der Vergleich der Docker-Alternativen erfolgte anhand einer positiven Gewichtung (Anzahl $+$) auf einer Skala von 1 bis 4. Dies ist für den groben Überblick ausreichend, bildet dabei aber keine detailliertere Gewichtung der Aspekte ab. Zur Abschätzung der Vor- und Nachteile reicht dieser grobe Vergleich jedoch aus. Die Gewichtung der Aspekte untereinander ist außerdem inhomogen, so ist die Anforderung der Datensparsamkeit z.B. wichtiger als die Geschwindigkeit. Daher sind die im folgenden näher erläuterten Punkte gesondert zu betrachten, da die Betrachtung der Summe über die Teilaspekte zum einen wenig aussagekräftig ist und zum anderen keine Form der Gewichtung der Vergleichskriterien beinhaltet.

\begin{itemize}
    \item Beispiel - ein charakteristischer Befehl/Referenz, siehe \url{https://github.com/hubotio/hubot/tree/master/docs/deploying}
    \item Geschwindigkeit - relative Zeit zur Ausführung der Anwendung
    \item Datensparsamkeit - Menge der ausgehenden Daten (weniger ist besser)
    \item Portierbarkeit - Grad der einfachen Portierung auf andere Systeme/Plattformen
    \item Persistenz - Möglichkeiten zur dauerhaften Datenspeicherung
\end{itemize}

Neben den positiven Kriterien, die sich untereinander nur qualitativ unterscheiden lassen, bestehen auch Ausschlusskriterien. Diese wurden hinsichtlich des Docker-Einsatzes \enquote{optimiert} und sind daher in anderen Kontexten kritisch zu betrachten. Die in diesem Projekt zu erfüllenden Aufgaben werden aber durch den Einsatz Dockers zu 100\% erfüllt.

\subsubsection{Entwicklungsumgebung mit Docker}

Basierend auf \autoref{img:deployment} ist die Systemumgebung für den Betrieb Hubots mindestens aus den Komponenten \enquote{Chatbot} und \enquote{Termindatenbank} aufgebaut. Aus Gründen der Wartbarkeit wird diese Trennung auch von den Docker-Containern aufrechterhalten.

Besondere Beachtung erfordert dabei die Verwaltung persistenter Daten, insbesondere der Datenbank. Da Docker-Container Daten standardmäßig nur volatil speichern\footnote{zur Gewährleistung der Idempotenz}, wird für den Erhalt der Datenbank ein persistentes Docker-Volume benötigt. Dafür wird ein \enquote{Volume} zwischen Hostsystem und Docker-Container eingebunden, so dass die Datenbank auch nach Beenden des Containers zur Verfügung steht.

Hinzu kommt ein weiterer Container zum Betrieb des eigentlichen Chatbots mittels NodeJS. Dieser wird mit dem gleichen virtuellen Netzwerk wie der DB-Container verbunden um Zugriff auf die Datenbank zu erhalten.

Weitere Einstellungen wie z.B. die zu öffnenden Ports folgen dem Paradigma \enquote{Convention over Configuration} (siehe \cite{NicholasChenConventionConfiguration2006}). Die Verbindung der Container untereinander erfolgt mittels docker-compose, in der docker-compose-Datei sind auch \textit{sinnvolle} Standardwerte enthalten, die schnell zum produktiven Einsatz führen (\cite{DockerCompose2018} und \autoref{lst:docker-compose}).
Dynamische Konfigurationen wie z.B. die UserID des Slackbots werden über entsprechende Umgebunsgvariablen in einer \verb+.env+-Datei übergeben und können problemlos während der Laufzeit geändert werden.

\todo{docker-compose in den Anhang}
% docker-compose beschreiben


%hubot-docker bla


\subsection{Slackidentifizierung}

% anflanschung der slacknutzer-tabelle // vielleicht in eine separate db??? oder reicht ne tabelle / viele tabellen für viele plattformen aus

\subsection{Botrealisierung}
\subsubsection{Ereignisgesteuerte Aktionen}
\subsubsection{Zeitgesteuerte Aktionen}


%name Bob, warum dieser? - vielleicht besser in die analyse/modell

\todo{Wie das mit den regexen - wegen ebnf recht einfach machbar}

%Anhang mit botcode dranheften

%wie erweiterbarkeit realisiert? - include module bla, file x
