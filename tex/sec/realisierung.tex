\subsection{Datenbank}

MySQL ausgewählt

Für die Erstellung der Datenbank entschieden sich die Autoren für die Verwendung eines Hilfswerkzeugs, in dem das Modell abgebildet und eine Befehlskette ausgegeben wird. Dadurch sollte eine Abweichung der Realisierung vom Modell vermieden und eine rasche prototypische Verifizierung ermöglicht werden. Aktuell (März 2018) existiert kein anerkannter Standard (wie z.B. UML) zur Beschreibung von Datenbanken, weshalb sich die Autoren für das Hilfswerkzeug \enquote{Vertabelo} entschieden. Dieses gibt die notwendigen SQL-Befehle aus, welche direkt für die Erstellung der Datenbank verwendet werden können. Weiterhin können die unter \url{vertabelo.com} angelegten Datenbankmodelle von mehreren Nutzern verwendet werden, was insbesondere für die geplante Weiterverwendung der Datenbank nützlich sein kann.
Die bereits in \autoref{img:db-schema} gezeigte Datenbank ist unter folgender URL abrufbar: \url{https://my.vertabelo.com/public-model-view/5BaPS6sRRrFicciuYZnBT1di8UbnI3fgTxZQEbkQsipFzy1ILjZG6Ek3evfFgV7Q}. Der daraus generierte SQL-Code befindet sich im Anhang.





\subsection{Containerisierung}
Docker bla

compose

mysql-docker warum den und nicht einen anderen

hubot-docker bla
\todo{weiter}



\subsection{Slackidentifizierung}

anflanschung der slacknutzer-tabelle // vielleicht in eine separate db??? oder reicht ne tabelle / viele tabellen für viele plattformen aus
\todo{weiter}

\subsection{Botrealisierung}

Bob, warum

Wie das mit den regexen - wegen ebnf recht einfach machbar

Anhang mit botcode dranheften

wie erweiterbarkeit realisiert? - include module bla, file x
\todo{weiter}
