\subsection{Anforderungen des Auftraggebers}
Der Auftraggeber teilt seine Anforderungen in notwendige und optionale Ziele ein, welche nachfolgend aufgelistet werden. Die notwendigen Ziele  für den Chatbot sind folgende:

\begin{itemize}
	\item in Slack verwendbar
	\item Datenbank, welche Veranstaltungen und Zuweisungen der Mitglieder enthält
	\item Erfassung, welche Mitglieder wie viele Dienste machen
	\item Abfragefunktion vorhanden
	\item zeitgesteuerte, ereignisbasierte und manuelle Erinnerungsfunktion
	\item Einschränkung der Zielgruppe möglich
	\item einfache Bedienbarkeit
	\item Erweiterbarkeit
	\item MIT-ähnliche Lizenzen für Drittanbietersoftware und -quelltext
	\item Dokumentation der Datenbank
	\item Unabhängkeit der relevanten Daten vom verwendeten Bot
	\item Zukunftssicherheit
\end{itemize}


optionale Ziele
\begin{itemize}
	\item aus iCal Termine extrahieren
	\item Dienstplan aus Doodle-ähnlicher Umfrage erstellen
	\item Nutzerberechtigungen
	\item Steuerung über E-Mail
	\item containerbasierte Lösung
	\item Caching für schnellere Abfragen
	\item Backup-Strategie
\end{itemize}

% Hier schon das Usecase-Diagramm hin?

\subsection{Festlegungen}

% Synonyme, Beschreibungen, Abgrenzungen für die Arbeit
