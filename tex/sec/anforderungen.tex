\label{anforderungen}
\subsection{Anforderungen des Auftraggebers}
Der Auftraggeber teilt seine Anforderungen in notwendige und optionale Ziele ein, welche nachfolgend aufgelistet werden.

\begin{itemize}
	\item Chatbot ist in Slack verwendbar
	\item Datenbank, für Veranstaltungen und Zuweisungen der Mitglieder
	\item Erfassung, welche Mitglieder wie viele Dienste gemacht haben
	\item Abfragefunktion für Termine
	\item zeitgesteuerte, ereignisbasierte und manuelle Erinnerungsfunktion
	\item Zielgruppe für Erinnerungen
	\item einfache Bedienbarkeit
	\item Erweiterbarkeit des Projekts
	\item MIT-ähnliche Lizenzen für Drittanbietersoftware und -quelltext
	\item Dokumentation der Datenbank
	\item Unabhängigkeit der relevanten Daten vom verwendeten Bot
	\item Zukunftssicherheit
\end{itemize}


Zusätzlich wurden folgende optionale Ziele festgelegt.
\begin{itemize}
	\item aus iCal Termine extrahieren
	\item Dienstplan aus Doodle-ähnlicher Umfrage erstellen
	\item Nutzerberechtigungen
	\item Steuerung über E-Mail
	\item containerbasierte Lösung
	\item Caching für schnellere Abfragen
	\item Backup-Strategie
\end{itemize}

% Hier schon das Usecase-Diagramm hin? - NEIN, hier kommt nur das rein, was Ferdi gesagt hat. Alles was wir selbst erarbeitet haben kommt danach.

\subsection{Festlegungen}
% Synonyme, Beschreibungen, Abgrenzungen für die Arbeit

Eine Menge an Terminen wird mit $T$ bezeichnet, Veranstaltungen mit $V$ und Sitzungen mit $S$.

Alle im Rahmen dieser Arbeit erstellten UML-Diagramme entsprechen dem aktuellen UML-Standard, Version 2.5.1. Dieser steht unter \url{https://www.omg.org/spec/UML/2.5.1/} zur Verfügung.

Alle das System nutzenden Akteure werden im Folgenden als Nutzer bezeichnet. Mitarbeiter sind alle Mitarbeiter des Studentenclubs Stecker. Administratoren sind spezielle Nutzer mit zusätzlichen Berechtigungen.
Es bestehen folgende Abhängigkeiten: 

$Nutzer = Administrator \cup Mitarbeiter$ 
%und $Administrator \subset Nutzer$ // unnötig

