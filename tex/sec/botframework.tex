
\subsection{Vergleichsparameter}
Für einen Vergleich der einzusetzenden Bot-Frameworks werden für diese im Folgenden Vergleichsparameter definiert. Dabei ist es wichtig, diese an den Anforderungen aus \autoref{anforderungen} sowie technischen Aspekten zu orientieren.

Jeder Parameter muss folgende Kriterien erfüllen:

\begin{itemize}
    \item messbar, als Zahlen-, Text- oder Wahrheitswert
    \item generisch, d.h. auf die zu vergleichenden Frameworks anwendbar
    \item nachweisbar, d.h. die Herkunft des Wertes ist nachvollziehbar
\end{itemize}

Aus \autoref{tab:botparameters} sind passende Bewertungsparameter gemäß der oben genannten Anforderungen zu entnehmen.

% evtl. booktabs statt tabularx
\begin{table}[htbp]
    \begin{tabularx}{\textwidth}{|l|X|p{6cm}|}
   \hline
   \textbf{Parameter} & \textbf{Beschreibung} & \textbf{Quelle} \\
   \hline
   Alter & Datum des ersten Commits im Master-branch & \verb+git log --reverse | head -3+\\
   \hline
   Aktivität & Commits in den Master-branch, Anz.\ Pull-Requests & GitHub Projektübersicht \\
   \hline
   Popularität & Anz.\ \enquote{Stars} und \enquote{Forks} & GitHub \\
    \hline
   Codequalität & Testabdeckung und Anz.\ Issues & GitHub, Tests vorhanden \\
   \hline
   Komplexität & Anz.\ Klassen, SLOC, Dateigröße ohne Beispiele, externe Abhängigkeiten etc. & \verb+du+, Code, \verb+git ls-files | xargs wc -l+ \\
   \hline
   Programmiersprache & primär verwendete Programmiersprache & GitHub-Seite \\
   \hline
   Integrationsgrad & Anz.\ bereits vorhandener Integrationen & Website, Dokumentation, npm \\
   \hline
   Erweiterbarkeit & Anz. der Erweiterungsmodule & Code, Dokumentation \\
   \hline
   Dokumentation & Dokumentation vorhanden ja/nein & docs-Ordner oder GitHub-Wiki \\
   \hline
   Lizenz & Art der Lizenz des Projektes & LICENSE.md im Git-Repositorium \\
   \hline
\end{tabularx}
\caption{Parameter zum Vergleich der Bot-Frameworks}
\label{tab:botparameters}
\end{table}

Die Aussagefähigkeit der in \autoref{tab:botparameters} enthaltenen Parameter ist stellenweise kritisch zu hinterfragen, da die Genauigkeit zugunsten der Generizität eingeschränkt wurde. 
Beispiele hierfür sind folgende:
\begin{itemize}
    \item Testabdeckung: je nach Auswahl und Integration der Testbibliothek kann die Testabdeckung immer 100\% betragen
\item SLOC\footnote{SLOC: Single Lines of Code}: viele Codezeilen deuten nicht zwingend auf guten Code hin (von Kommentaren wird hier -- per Definition -- abgesehen)
\end{itemize}

Unter Beachtung der Einschränkungen und der Kombination aller Parameter ist eine differenzierte Gegenüberstellung durchzuführen.

\subsection{Auswahl möglicher Frameworks}

Aufbauend auf den in \autoref{anforderungen} gestellten Anforderungen entfallen Frameworks:
\begin{itemize}
    \item deren Quelltext nicht frei verfügbar ist
    \item welche eine dauerhafte Verbindung zum Hersteller benötigen
    \item nicht kostenfrei sind
\end{itemize}

Durch diese Einschränkungen sind z.B. das von Facebook entwickelte wit-Botframework (\cite{FacebookWitai2013})
und das von Microsoft entwickelte Botframework (\cite{MicrosoftMicrosoftBotFramework2016}) nicht Teil genauerer Betrachtungen.
Eine weitere Gruppe von Botframeworks zeichnet sich durch eine (meist zwingende) Anbindung an eine Spracherkennung aus, die für den hier gewünschten Anwendungsfall nicht benötigt wird. Dadurch werden Frameworks wie beispielsweise Dialogflow (\cite{DialogflowDialogflow2018}) nicht Teil weiterer Betrachtungen sein.


Weitere Botframeworks sind unter \cite{AbdelhaiawesomebotsAwesomeLinks2018} aufgelistet. 
Von den dort aufgelisteten Frameworks entsprechen Botkit (\cite{BotkitBotkitToolkitbuilding2017}) und Hubot (\cite{GitHubGettingStartedHubot2009})  den Anforderungen.

\subsection{Vergleich der Frameworks}

Aufbauend auf den beiden vorherigen Unterpunkten wurde ein möglichst neutraler Vergleich der beiden Frameworks \textit{Hubot} und \textit{Botkit} ermöglicht, siehe \autoref{tab:comparebotkithubot}.

\begin{table}[htbp]
    \centering
    \begin{tabularx}{\textwidth}{|l|X|X|}
        \hline
        \textbf{Parameter} & \textbf{Botkit} & \textbf{Hubot} \\
        \hline
        Alter (Stand 02.2018) & $\approx$ 2 Jahre & $\approx$ 5 Jahre \\
        \hline
        Aktivität & \makecell[l]{2078 Commits\\ 29 Pull-Requests} & \makecell[l]{2011 Commits\\ 5 Pull-Requests} \\
        \hline
        Popularität & \makecell[l]{7.813 Sterne\\ 1.714 Forks} & \makecell[l]{13.817 Sterne\\ 3.269 Forks} \\
        \hline
        Codequalität & \makecell[l]{115 Issues\\ Testabdeckung 100\%} & \makecell[l]{30 Issues\\ Testabdeckung 100\%} \\
        \hline
        Komplexität & 35259 SLOC & 7472 SLOC \\
        \hline
        Programmiersprache & TypeScript & CoffeeScript \\
        \hline
        Integrationsgrad & 12 & 62 \\
        \hline
        Erweiterbarkeit & 28 & $\approx$ 50 \\
        \hline
        Dokumentation & ja & ja \\
        \hline
        Lizenz & frei, MIT & frei, MIT \\
        \hline
    \end{tabularx}
    \caption{Vergleich von Botkit und Hubot}
    \label{tab:comparebotkithubot}
\end{table}

% Zwischenfazit: Botkit ist größer und umfangreicher, Hubot kleiner und simpler.
% Würde ich am Ende von TypeScript vs. CoffeeScript abhängig machen

Wie in \autoref{tab:comparebotkithubot} ersichtlich, bietet Hubot aufgrund der längeren Entwicklungszeit eine größere Anzahl von Nutzern und Integrationen. Des Weiteren ist Hubot auf die praktische Notwendigkeit bei GitHub zurückzuführen, die Entwicklungsprozesse an ChatOps\footnote{Ausführen von Deployment-Befehlen etc. direkt aus dem Chat-Programm, Hintergrund siehe: \url{https://speakerdeck.com/jnewland/chatops-at-github}}

Im Gegensatz dazu bietet Botkit flexiblere Einsatz- und Integrationsmöglichkeiten in andere Dienste wie NLU\footnote{Natural language understanding}, Apps und Messenger. Aufgrund des relativ jungen Projektalters von zwei Jahren und der vergleichsweise hohen Anzahl an Bugtickets (\enquote{Issues}) und Codezeilen, wird die Codequalität von Botkit geringer als die von Hubot bewertet.

Beide Frameworks dienen dabei unterschiedlichen Zielen. Während Botkit eher als Rahmenwerk für darauf aufbauende Bots anzusehen ist, ist der Bot als solches direkt bei Hubot integriert und kann modular erweitert werden. Da beide Frameworks unter MIT-Lizenz stehen und zu JavaScript kompilieren, sind diese aus lizenzrechtlicher und technischer Sicht für das Projekt \enquote{Steckerbot} geeignet.

Für den in \autoref{anforderungen} beschriebenen Einsatzzweck fällt die Entscheidung auf \textbf{Hubot}. Im Vergleich zu Botkit bewerten die Autoren Hubot als strukturell simpler und näher am Ziel der praktischen Slack-Integration ausgerichtet. Botkit erfüllt diese Anforderungen auch, hatte jedoch keine so lange Reifephase wie Hubot.

Im weiteren Verlauf dieses Dokuments werden deshalb \enquote{Hubot}, \enquote{Botframework} und \enquote{Bot} synonym verwendet.

