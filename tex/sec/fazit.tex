Die Autoren konnten das derzeit geeignetste Botframework für den beschriebenen Anwendungsfall identifizieren. Weiterhin wurden, durch die Betrachtung der Verwendungsweise des Bots, die Kernbestandteile der Chatbotinteraktion sowie der notwendigen und einfach zu nutzenden Befehle erkannt und beschrieben. Zusätzlich wurden Beispiele für Sätze gegeben, welche in einer besseren sprachlichen Integrierung des Chatbots in den Konversationsverlauf resultieren. 

Es wurde eine Datenbank für die Terminverwaltung erstellt, welche einfach erweiterbar, nicht plattformabhängig und zukunftssicher ist. Durch die zeitgesteuerte Erinnerungsfunktion wurde ebenfalls das Ziel der automatischen Erinnerung der Nutzer an bevorstehende Termine erreicht. Als optionales Ziel konnte die containerbasierte Lösung durch Docker vollzogen werden, wodurch das Deployment im Produktiveinsatz auf bereits vorhandenen Systemen ohne großes Konfliktpotential möglich ist.

Es konnten somit alle notwendigen Anforderungen erreicht werden. Mit Ausnahme der begrenzten Zielgruppe für Erinnerungen wurde ebenfalls eine prototypische Realisierung bereitgestellt. Die Voraussetzungen hierfür wurden jedoch geschaffen, wie \autoref{sec:botx} beschreibt.
