Die Autoren konnten das derzeit geeignetste Botframework für den beschriebenen Anwendungsfall identifizieren. Weiterhin konnte durch die Betrachtung der Verwendungsweise des Bots die Kernbestandteile der Chatbotinteraktion, der notwendigen Befehle und einfach zu nutzender Befehle, welche einem Schema folgen, erkannt und beschrieben werden. Zusätzlich wurden Beispiele für Sätze gegeben, welche in einer besseren sprachlichen Integrierung des Chatbots in den Konversationsverlauf resultieren. Weiterhin konnte eine Datenbank für die Terminverwaltung erstellt werden, welche einfach erweiterbar, nicht Plattformabhängig und zukunftssicher ist. Durch die zeitgesteuerte Erinnerungsfunktion wurde ebenfalls das Ziel der automatischen Erinnerung der Nutzer an bevorstehende Termine erreicht. Als optionales Ziel konnte die containerbasierte Lösung durch Docker vollzogen werden, wodurch das Deployment im Produktiveinsatz auf bereits vorhandenen Systemen ohne großes Konfliktpotential möglich ist.