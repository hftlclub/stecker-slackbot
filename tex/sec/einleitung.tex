Dieses Dokument wurde im Rahmen des Masterprojekts des Studiengangs Informatik an der HTWK-Leipzig erstellt. Es beinhaltet die Entwicklung und prototypische Umsetzung eines Chatbots für die Terminverwaltung des Studentenclubs \glqq Stecker\grqq .

Der Auftrag wurde von der technischen Abteilung des Clubs gestellt. Die Motivation hierfür ist eine einfache, schnelle und zentrale Verwaltung von Schichtplänen, um den Vorstand in der Organisation der Veranstaltungen zu entlasten.
Es wird ein Chatbot benötigt, welcher über die Kommunikationsplattform \glqq Slack\grqq gesteuert werden kann und auch von technisch nicht versierten Mitgliedern bedienbar ist.

Dieses Dokument wird die Anforderungen an den Chatbot präsentieren und bereits vorhandene Lösungen bewerten und gegenüberstellen. Anschließend wird ein Modell für die Verwendung eines Frameworks im Zusammenhang mit Use-Cases erstellt. Weiterhin wird ein Datenbankmodell für die Terminverwaltung entworfen, welches sich nicht nur auf die bekannten Anwendungsfälle beschränken wird. Es wird gesondert auf die praktische Realisierung und die zu beachtenden Probleme bei der Einrichtung eingegangen. Das fertige System wird anschließend analysiert und bewertet. Im Schlusswort werden die Autoren das entworfene System den Anforderungen gegenüberstellen und abschließend eine Aussicht für die weitere Entwicklung geben.
