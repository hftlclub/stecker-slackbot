
\section{Einleitung}

\clearpage
\section{Hauptteil}

\clearpage
\section{Anforderungen und Festlegungen}
% Ferdi will, Ferdi verlangt, Ferdi lässt Spielraum
% Festgelegte Begriffe, allgemeingültige Aussagen für das Dokument

\clearpage
\section{Auswahl eines Bot-Frameworks}
% HuBot, BotKit
% Einschränkungen der Auswahl weil ...
\subsection{Vergleichsparameter}
Für einen Vergleich der einzusetzenden Bot-Frameworks werden für diese im Folgenden Vergleichsparameter definiert. Dabei ist es wichtig, diese an den Anforderungen aus dem vorherigen Kapitel sowie technischen Aspekten zu orientieren.

Jeder Parameter muss folgende Kriterien erfüllen:

\begin{itemize}
    \item messbar, als Zahlen- Text- oder Wahrheitswert
    \item generisch, d.h. auf die zu vergleichenden Frameworks anwendbar
    \item nachweisbar, d.h. die Herkunft des Wertes ist nachvollziehbar
\end{itemize}

Aus \autoref{tab:botparameters} sind passende Bewertungsparameter gemäß der oben genannten Anforderungen zu entnehmen.

% evtl. booktabs statt tabularx
\begin{table}[htbp]
    \begin{tabularx}{\textwidth}{lXp{6cm}}
   \hline
   \textbf{Parameter} & \textbf{Beschreibung} & \textbf{Quelle} \\
   \hline
   Alter & Datum des ersten Commits im Master-branch & \verb+git log --reverse | head -3+\\
   \hline
   Aktivität & Commits in den Master-branch, Anz. Pull-Requests & GitHub Projektübersicht \\
   \hline
   Codequalität & Testabdeckung und Anz. Issues & GitHub, Tests vorhanden \\
   \hline
   Komplexität & Anz. Klassen, SLOC, Dateigröße \textbf{ohne Beispiele, externe Abhängigkeiten etc.} & \verb+du+, Code, \verb+git ls-files | xargs wc -l+ \\
   \hline
   Programmiersprache & primär verwendete Programmiersprache & GitHub-Seite \\
   \hline
   Integrationsgrad & Anz. bereits vorhandener Integrationen & Website, Dokumentation \\
   \hline
   Erweiterbarkeit & Anz. an Erweiterungsmodulen & Code, Dokumentation \\
   \hline
   Dokumentation & Dokumentation vorhanden ja/nein & docs-Ordner oder GitHub-Wiki \\
   \hline
   Lizenz & Art der Lizenz des Projektes & LICENSE.md im Git-Repositorium \\
   \hline
\end{tabularx}
\caption{Parameter zum Vergleich der Bot-Frameworks}
\label{tab:botparameters}
\end{table}

Die Aussagefähigkeit der in \autoref{tab:botparameters} enthaltenen Parameter ist stellenweise kritisch zu hinterfragen, da die Genauigkeit zugunsten der Generizität eingeschränkt wurde. 
Beispiele:
\begin{itemize}
    \item Testabdeckung: je nach Auswahl und Integration der Testbibliothek kann die Testabdeckung immer 100\% betragen
\item SLOC: viele Codezeilen deuten nicht zwingend auf guten Code hin (von Kommentaren wird hier abgesehen)
\end{itemize}

Unter Beachtung der Einschränkungen und der Kombination aller Parameter ist eine differenzierte Gegenüberstellung durchführbar.

\subsection{Auswahl möglicher Frameworks}

Aufbauend auf den zu Beginn gestellten Anforderungen entfallen Frameworks, die:
\begin{itemize}
    \item deren Quelltext nicht frei verfügbar ist
    \item eine dauerhafte Verbindung zum Hersteller benötigen
    \item nicht kostenfrei sind
\end{itemize}

Durch diese Einschränkungen sind z.B. das von facebook entwickelte wit-Botframework (\url{https://wit.ai/} und das von Microsoft entwickelte Bot-framework (\url{https://dev.botframework.com/}) nicht Teil weiterer Betrachtungen.
Eine weitere Gruppe von Botframeworks zeichnet sich durch eine (meist zwingende) Anbindung an eine Spracherkennung aus, die für den hier gewünschten Anwendungsfall nicht benötigt wird. Dadurch wird z.B. \url{api.ai} nicht Teil weiterer Betrachtungen sein.


Weitere Botframeworks sind unter \url{https://github.com/abdelhai/awesome-bots} aufgelistet. 
Von den dort aufgelisteten Frameworks entsprechen hier \textbf{Botkit} und \textbf{Hubot} den Anforderungen.

\subsection{Vergleich der Frameworks}

\begin{table}[htbp]
    \centering
    \begin{tabularx}{\textwidth}{lXX}
        \hline
        \textbf{Parameter} & \textbf{Botkit} & \textbf{Hubot} \\
        \hline
        Alter (Stand 02.2018) & ~2 Jahre & ~5 Jahre \\
        \hline
        Aktivität & 2078 Commits, 29 Pull-Requests & 2011 Commits, 5 Pull-Requests \\
        \hline
        Codequalität & 115 Issues, Testabdeckung 100\% & 30 Issues, Testabdeckung 100\% \\
        \hline
        Komplexität & 35259 SLOC & 7472 SLOC\\
        \hline
        Programmiersprache & TypeScript & CoffeeScript\\
        \hline
        Integrationsgrad & tbd & tbd \\
        \hline
        Erweiterbarkeit & tbd & tbd \\
        \hline
        Dokumentation & ja & ja \\
        \hline
        Lizenz & frei, MIT & frei, MIT \\
    \end{tabularx}
    \caption{Vergleich von Botkit und Hubot}
    \label{tab:comparebotkithubot}
\end{table}

% Zwischenfazit: Botkit ist größer und umfangreicher, Hubot kleiner und simpler.
% Würde ich am Ende von TypeScript vs. CoffeeScript abhängig machen

\clearpage
\section{Entwurf eines Modells}
% Kommunikationspfade, Blockdiagramme, UML
% Datenhaltung, Datentransport, Sicherheit, Zukunftssicherheit, Austauschbarkeit, Robustheit, Erweiterbarkeit

\clearpage
\section{Realisierung des Modells}
% Warum welche Sprache
% Warum Abkürzungen im ggs zu Modell genommen
% Welche praktischen Hürden überwunden

\clearpage
\section{Analyse}
% Alle Anforderungen erfüllt?
% Wie gemessen?
% Blockdiagramme, MSCs, Fehlerbetrachtung
% Performance - Latenz?

\clearpage
\section{Fazit}
% Funzt? Jo
% Was wurde erledigt, was ¬ und warum?

\clearpage
\section{Aussicht}
% Empfehlungen für Verbesserungen, Erweiterungsmöglichkeiten, ...

\clearpage
\section{Benutzung des Endprodukts}
% Wo Daten eingeben, Wie per Hand testen, Wie Komponenten verbinden, (minimal) Beispiel


%Literatur
\nocite{GitHubGettingStartedHubot}
\nocite{BotkitBotkitToolkitbuilding}
\nocite{SlackConversationsAPI}
